\chapter{Système global}
Après avoir décrit chaque bloc de façon individuelle, il est nécessaire d'observer le système dans son ensemble afin de pouvoir valider, ou non, son bon fonctionnement. 

\section{Assemblage}
Il est important que chacun des blocs se comportent au sein
du système global comme défini dans les sections précédentes, 
autrement dit : que les blocs n'interfèrent pas entre eux.
Pour cela, il est possible d'ajouter des amplificateurs 
opérationnels en suiveur\footnote{Idéalement, les amplificateurs opérationnels
ont une impédance d'entrée infinie et une impédance de sortie nulle.} 
entre les blocs afin d'assurer une liaison imperméable. 
%Le premier se trouve entre le filtre sinusoïdale
%et le sigma-delta. En effet, la résistance équivalente du
%sigma-delta n'est pas infini et interfère donc avec le filtre
%si il ne sont pas isolés à l'aide d'un suiveur.
%Les autres blocs ne nécessitent pas d'étage tampon en vue
%de leurs agencements individuels

\section{Validation}
Afin de valider le circuit de façon globale le tableau ci
dessous met en évidence l'erreur en fréquence et le taux de
distortion harmonique observés à la sortie du filtre pour
diverses notes jouées.La validation du sigma-delta est
effectuée grâce à une application \textsc{Android} permettant de
mesurer la fréquence d'un signal audio. Ceci a été effectué
pour l'ensemble des notes, mais le tableaux ne reprend que
quelques valeurs significatives et les maximas. Le clavier a
été calibré avant la prise de mesure.

\begin{center}
	\begin{tabular}{|c|c|c|c|}
		\hline
		Clavier &  \multicolumn{2}{c|}{Filtre} &  Sortie audio \\
		 \hline
		Note &  Erreur fréquence (\%) & TDH (\%) &  Erreur fréquence (\%) \\
		\hline
		D5 &  -4.1 & 1.55 & -3.8\\
		B5 &  2.9 & 2.26 & 3.2\\
		F6 &  1.6 & 1.92 & 1.3\\
		F7 &  0.9 & 2.02 & 1.5\\
		C8 &  0.3 & 2.08 & 0.2\\
		F8 &  2.3 & 1.29 & 2.1\\
		\hline
	\end{tabular}
\end{center}

Cette partie du système remplit bien son rôle. L'erreur en 
fréquence se trouve dans une gamme raisonnable (de -3.8\% à 3.2\%)
sur l'ensemble de la gamme fréquentielle couverte. Les notes 
correspondent donc chaque fois à la note souhaitée, et non à celle
supérieure ou inférieure.
Quant au taux de distorsion harmonique, il ne dépasse pas 2.8\%.

Des 'glitchs' apparaissent sur l'ensemble du circuit et ce à 
même fréquence que la fréquence de sortie du sigma-delta. Ceci est dû au fait que
le sigma delta pompe du courant à la source par à coup. Cela ne modifie
que légèrement la qualité du système.

\section{Pistes d'amélioration}
\label{sec:pistes-amelioration}
La conception et la mise en œuvre de chaque bloc ont été effectuées
de manière locale et non globale, des blocs
intermédiaires (adaptateurs) ont dû être ajoutés en conséquence. Une vision plus
globale dès le début du projet aurait permis une simplification du
système final ainsi qu'une diminution du nombre de composants et
donc de puissance utilisée et d'erreurs possibles.

En pratique, il aurait été possible de modifier la gamme de sortie
du VCO afin que celle-ci corresponde à l'entrée du filtre (
triangle entre \unit{-3}{\volt} et \unit{+3}{\volt}), et donc 
supprimer l'adaptateur. Il faut pour cela modifier les seuils
de basculement de la bascule à hystérèse afin qu'ils soient de
\unit{-3}{\volt} et \unit{+3}{\volt}, et ensuite adapter la
constante d'intégration pour que le lien amplitude d'entrée/fréquence
de sortie soit toujours respecté. D'un point de vue circuit, il s'agit
donc de changer les résistances caractéristiques de la bascule, et le
condensateur nécessaires à l'intégration. Réaliser l'opération inverse,
c'est-à-dire modifier les caractéristiques du filtre afin qu'il accepte
en entrée un signal correspondant à la sortie actuelle du VCO 
(triangle entre \unit{-1}{\volt} et \unit{0}{\volt}) n'aurait pas été 
possible étant donné la conception de celui-ci (voir section \ref{sec:filtre}). 
Pour cela, il est nécessaire de concevoir un nouveau filtre.

Il est également possible de réaliser un clavier dont la sortie serait directement
liée à la masse lorsqu'aucune touche n'est enfoncée, sans nécessiter un comparateur
et un switch. Cependant cette solution nous échappe encore.

Comme mis en avant dans la section \ref{sec:sigma-delta} concernant le
sigma-delta, le bloc 'sigma-delta' actuel n'utilise qu'une mince zone
de fréquence de commutation. En effet, il serait possible de diminuer
les racines de la parabole de la figure \ref{fig:sigma-delta-lim}, 
afin que la zone de fréquence atteignable pour le signal d'entrée 
(compris entre \unit{-3}{\volt} et \unit{3}{\volt})
soit plus importante tout en s'assurant que la fréquence minimale
atteinte soit au-delà de la fréquence de Shannon (\unit{16}{\kilo\hertz} dans
notre cas). Il suffit de fixer judicieusement les deux tensions d'entrée qui produisent
en sortie un signal dont la fréquence correspond à la fréquence de Shannon. 
Ces points pourraient être fixés à \unit{-4}{\volt} et \unit{4}{\volt}
afin de garder une marge d'erreur. La contrainte sur la fréquence
de sortie maximale reste identique.
Ces trois points donne l'équation de la nouvelle parabole. Une plus 
grande zone de fréquence de commutation permet une meilleure reconstitution
du signal d'entrée.

Concernant les 'glitchs' observés, l'unique solution est d'utiliser deux 
sources d'alimentation distinctes : une pour le sigma-delta, et une pour le reste du circuit.