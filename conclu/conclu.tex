\chapter{Conclusion}
Au terme de ce quatrième quadrimestre, c'est avec un sentiment de satisfaction
que nous clôturons notre projet. En fabriquant un synthétiseur analogique,
nous avons appris plein de choses et développé de nouvelles compétences.
Tout d'abord, partant de la décomposition du circuit en blocs fonctionnels,
nous avons traduit ces blocs en circuits électriques que nous avons simulé.
Par la suite, nous les avons montés et testés. Et pour finir, nous avons confronté
la théorie, les simulations et la réalité. Nous avons donc acquis la démarche de
conception de circuits électriques.

Le son produit par notre synthétiseur est très proche d'une sinusoïde. Le taux de
distorsion harmonique est au maximum de 3\%. L'erreur maximale sur la fréquence
d'une note est de l'ordre de 3\%. Cette erreur est due aux incertitudes sur les
valeurs des composants utilisés dans le circuit.
L'amplitude du signal de sortie varie très peu sur l'ensemble de la gamme de sortie.
Notre synthétiseur permet de jouer des notes de l'octave 5 à l'octave 8. 
Nous couvrons donc 4 octaves.

La vision par blocs du synthétiseur n'est pas assez globale. Cela nous posa
quelques petits problèmes lors de l'assemblage. Pour résoudre ces problèmes, 
nous avons dû ajouter des adaptateurs entre chaque bloc fonctionnel. Une piste
d'amélioration est donc d'envisager la conception de façon plus globale afin
de ne plus avoir besoin de ces adaptateurs. Une autre piste d'amélioration intéressante
est de réduire les perturbations introduite par le MLI sur l'ensemble du signal en 
l'alimentant avec une autre source que le reste du système.