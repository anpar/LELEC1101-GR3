\chapter{Introduction}
La réalisation d'un synthétiseur analogique permet l'application
de l'ensemble des connaissances acquises en électricité au cours
de nos deux premières années d'études : conception de blocs fonctionnels, 
modélisation du fonctionnement de circuits électriques et réalisation de ceux-ci,
utilisation de logiciel de simulation tel que LT-Spice et interprétation des résultats réels. \\

Le rapport est organisé de la façon suivante. Il y sera d'abord expliqué
le fonctionnement général d'un synthétiseur analogique. Les 
différents blocs qui composent le synthétiseur seront ensuite
analysés séparément en décrivant pour chacun d'entre eux le
fonctionnement théorique du bloc, son dimensionnement et sa
mise en place en circuit réel, et la confrontation des mesures
réalisées avec les attentes théoriques. Un passage validera ensuite le système globale obtenu,
et proposera différentes pistes d'amélioration.