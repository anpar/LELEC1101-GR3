% little trick to replace lib.tex by this
\renewcommand{\doctitle}[1]{
	\section{#1}
}
\doctitle{Dimensionnement du modulateur sigma-delta}

\subsection{Fonctionnement et théorie}
Le schéma bloc du modulateur sigma-delta se trouve
à la figure \ref{fig:sigma-delta-schema-blocs}. Ici,
on choisit une bascule asymétrique.

\begin{figure}[ht]
	\centering
	\includegraphics[scale=0.75]{img/schema-blocs.png}
	\caption{Schéma bloc du modulateur sigma-delta.}
	\label{fig:sigma-delta-schema-blocs}
\end{figure}

Dans un premier temps, calculons la période
d'oscillation de la sortie (qui est identique
à la période d'oscillation de $V_I$ sur la figure
\ref{fig:sigma-delta-schema-blocs}.

On démarre avec un signal $V_{\text{ref}}$ positif et
$V_{\text{out}} = 0$. $V_I$ est alors immédiatement positif
et $V_{\text{out}}$ sature directement à $E$. Comme $V_{\text{ref}}$
est $\leq E$, $V_I$ va maintenant décroître jusqu'à atteindre
$\Delta V$. A ce moment précis, on aura à nouveau $V_{\text{out}} = 0$
et donc $V_I$ va croître jusqu'à atteindre 0, et ainsi de suite.

Sur base de cela, on peut facilement calculer
le temps de descente $t_f$ et le temps de montée $t_r$
du signal $V_I$\footnote{Ce signal sera soit un signal triangulaire,
soit un signal en dents de scie, selon la valeur de $V_{\text{ref}}$.}.
On trouve facilement,
\[ t_f = -\frac{\Delta V}{(V_{\text{ref}} - E)K},\]
\[ t_r = \frac{\Delta V}{KV_{\text{ref}}}.\]
La période $T$ étant la somme du temps de descente et du temps
de montée, on trouve
\[ T = \frac{\Delta V}{K}\left(\frac{1}{V_{\text{ref}}} - \frac{1}{V_{\text{ref}} - E}\right)\]
et donc finalement
\begin{equation} 
	f = -\frac{K}{\Delta V} \frac{V_{\text{ref}}(V_{\text{ref}}-E)}{E}.
	\label{eq:sigma-delta-frequency}
\end{equation}

\paragraph{Remarque}
A partir du temps de descente et du temps de montée, on
peut prouver que la moyenne du signal carré $V_{\text{out}}$
vaut bien $V_{\text{ref}}$. Il suffit de démontrer l'égalité
suivante
\[ \frac{E \cdot t_f + 0 \cdot t_r}{T} = V_{\text{ref}}.\]

La fréquence en fonction de $V_{\text{ref}}$ est donc
une parabole avec une racine en \unit{0}{\volt} et une
racine en \unit{E}{\volt}.

On peut déduire plusieurs chose de l'équation \ref{eq:sigma-delta-frequency}. 
Premièrement, la fréquence de sortie maximale est atteinte pour 
$V_{\text{ref}} = \frac{E}{2}$ et vaut
\[ f_{\text{max}} = \frac{K}{\Delta V}\frac{E}{4}. \]
Ensuite, pour un signal $V_{\text{ref}}$ sinusoïdal dont
l'amplitude peut être négative, la fréquence sature.
Or, dans le cas de notre synthétiseur, le signal $V_{\text{ref}}$
est la sortie de notre VCO (après passage dans un filtre pour
en extraire une sinusoïdale pure).
Il faudra donc ``déplacer'' la parabole de manière à ce qu'elle soit
centré autour de l'origine.

\subsection{Dimensionnement et circuit réel}
Le circuit du modulateur sigma-delta est représenté
à la figure \ref{fig:sigma-delta-circuit}.

\begin{figure}[ht]
	\centering
	\includegraphics[scale=0.7]{img/sigma-delta-circuit.png}
	\caption{Circuit du modulateur.}
	\label{fig:sigma-delta-circuit}
\end{figure}

On va résoudre ce circuit pour obtenir des équations
de la même forme que celles de la figure
\ref{fig:sigma-delta-schema-blocs}.
On se concentre d'abord sur l'amplificateur
opérationnel. Grâce à la boucle de contre
réaction négative, on peut dire $v_- = v_+ = 0$.
On peut ensuite obtenir les courants suivants
\[ i_{R_4} = \frac{V_{\text{ref}}}{R_4},\]
\[ i_{R_6} = \frac{V_{\text{ee}}}{R_6},\]
\[ i_{R_3} = \frac{V_{\text{out}}}{R_3},\]
\[ i_{C_1} = -C_1\fdif{v_{\text{in}}}{t}.\]
On applique ensuite KCL et on écrit
\[ i_{C_1} = i_{R_4} + i_{R_6} + i_{R_3}.\]
De cette relation, on tire
\[ v_{\text{in}} = -\frac{1}{C_1}\int \frac{V_{\text{ref}}}{R_4}
+ \frac{V_{\text{ee}}}{R_6} + \frac{V_{\text{out}}}{R_3}.\]
Pour se ramener à l'équation de la figure
\ref{fig:sigma-delta-schema-blocs}, on pose
$V'_{\text{ref}} = -R_3(\frac{V_{\text{ref}}}{R_4}+\frac{V_{\text{ee}}}{R_6})$
pour enfin obtenir
\[ v_{\text{in}} = \frac{1}{C_1R_3} \int V'_{\text{ref}} - V_{\text{out}}\]
où $V'_{\text{ref}}$ correspond au $V_{\text{ref}}$
de la figure \ref{fig:sigma-delta-schema-blocs}.

Pour dimensionner le modulateur, on doit respecter plusieurs
contraintes. Premièrement la fréquence pour $V_{\text{ref}} =$
\unit{7.5}{\volt} doit être de \unit{80}{\kilo\hertz}. Et deuxièment,
on doit déplacer la parabole
de manière à ce que ces racines soient \unit{-15}{\volt} et
\unit{+15}{\volt}. Enfin, on doit choisir $\Delta V$ de manière
à ce que la bascule ne soit pas sensible au bruit (quelques millivolts).

On va directement anticiper une non-idéalité de la bascule,
la valeur de saturation $E$ n'est pas égale à la tension
d'alimention. On a plutôt $E \approx$ \unit{13.5}{\volt}.

Pour centrer la parabole, il faut que $\frac{R_3}{R_6}V_{\text{ee}}$
soit égale à \unit{6.75}{\volt}. Il faut ensuite étirer la
parabole de manière à ce que ses racines soient $\pm$\unit{15}{\volt}.
Il faut donc $\frac{R_3}{R_4} = 0.45$. 

En utilisant des valeurs de composants standards (série de Renard E12). 
On peut choisir, $R_3 =$ \unit{22}{\kilo\ohm} et $R_4 = R_6 =
$ \unit{48.5}{\kilo\ohm}.

On passe ensuite à la contrainte sur la fréquence. On a comme
relation 
\[ \frac{K}{\Delta V}\frac{E}{4} = 80000.\]
On peut fixer arbitrairemet $\Delta V$ à \unit{1}{\volt}. On a alors
$K = \frac{1}{C_1R_3} = 23703.7037$ et donc $C1 =$ \unit{1.9}{\nano\farad}.
Enfin, comme $\Delta V = \frac{R_1}{R_2}E$, on peut par exemple
choisir $R_1 =$ \unit{10}{\kilo\ohm} et $R_2 =$ \unit{134.6}{\kilo\ohm}.

% TODO : on peut sans doute réaliser encore un meilleur
% dimensionnement en considérant dans les calculs les valeurs
% exactes (je veux dire par là les valeurs mesurées sur circuit,
% pas les valeurs théoriques).

Pour appliquer la signal de sortie du modulateur
à l'étage suivant du circuit, il faudra utiliser un diviseur
résistif car l'étage suivant ne supporte pas des entrées supérieures
à \unit{5}{\volt}.

\subsection{Confrontation des mesures et de la théorie}
En superposant le graphe théorique que l'on peut obtenir avec les valeurs
obtenues dans la section précédentes
et des mesures effectués sur une implémentation en circuit
du modulateur, on obtient la figure \ref{fig:sigma-delta-f-vs-vref-dim-vs-real}.

\begin{figure}[ht]
	\centering
	\includegraphics[scale=0.7]{img/sigma-delta-f-vs-vref-dim-vs-real.png}
	\caption{En bleu, les prévisions théoriques et en vert les mesures.}
	\label{fig:sigma-delta-f-vs-vref-dim-vs-real}
\end{figure}

On constate que la théorie colle assez bien à la réalité. Le
faible décalage dépend sans doute des tolérances des résistances,
des variations dans les alimentations (le MyDAQ sort, dans ce cas, du
\unit{+14.10}{\volt} et du \unit{-14.62}{\volt} plutôt que du
$\pm$\unit{15}{\volt}), des variations dans la valeur de saturation
$E$ ($\approx$ \unit{13.62}{\volt}). On pourrait effectuer un
dimensionnement plus précis à partir de ces valeurs réelles afin
d'obtenir une prévision théorique encore plus proches de la réalités.
% FIXME (à confirmer)
Le problème, c'est que les valeurs des tensions
d'alimentations (par exemple) dépendent justement de la charge 
connectée, et donc du choix des résistances effectuées lors du dimensionnement.

% little trick to replace footer.tex by an empty file