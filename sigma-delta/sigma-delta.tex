% little trick to replace lib.tex by this
\renewcommand{\doctitle}[1]{
	\section{#1}
}
\doctitle{Dimensionnement du modulateur sigma-delta}

\section{Fonctionnement et théorie}
Le schéma bloc du modulateur sigma-delta se trouve
à la figure \ref{fig:sigma-delta-schema-blocs}. Ici,
on choisit une bascule asymétrique.

\begin{figure}[ht]
	\centering
	\includegraphics[scale=0.75]{img/schema-blocs.png}
	\caption{Schéma bloc du modulateur sigma-delta.}
	\label{fig:sigma-delta-schema-blocs}
\end{figure}

Dans un premier temps, calculons la période
d'oscillation de la sortie (qui est identique
à la période d'oscillation de $V_I$ sur la figure
\ref{fig:sigma-delta-schema-blocs}.

On démarre avec un signal $V_{\text{ref}}$ positif et
$V_{\text{out}} = 0$. $V_I$ est alors immédiatement positif
et $V_{\text{out}}$ sature directement à $E$. Comme $V_{\text{ref}}$
est $\leq E$, $V_I$ va maintenant décroître jusqu'à atteindre
$\Delta V$. A ce moment précis, on aura à nouveau $V_{\text{out}} = 0$
et donc $V_I$ va croître jusqu'à atteindre 0, et ainsi de suite.

Sur base de cela, on peut facilement calculer
le temps de descente $t_f$ et le temps de montée $t_r$
du signal $V_I$\footnote{Ce signal sera soit un signal triangulaire,
soit un signal en dents de scie, selon la valeur de $V_{\text{ref}}$.}.
On trouve facilement,
\[ t_f = -\frac{\Delta V}{(V_{\text{ref}} - E)K},\]
\[ t_r = \frac{\Delta V}{KV_{\text{ref}}}.\]
La période $T$ étant la somme du temps de descente et du temps
de montée, on trouve
\[ T = \frac{\Delta V}{K}\left(\frac{1}{V_{\text{ref}}} - \frac{1}{V_{\text{ref}} - E}\right)\]
et donc finalement
\[ f = -\frac{K}{\Delta V} \frac{V_{\text{ref}}(V_{\text{ref}}-E)}{E}.\]

\paragraph{Remarque}
A partir du temps de descente et du temps de montée, on
peut prouver que la moyenne du signal carré $V_{\text{out}}$
vaut bien $V_{\text{ref}}$. Il suffit de démontrer l'égalité
suivante
\[ \frac{E \cdot t_f + 0 \cdot t_r}{T} = V_{\text{ref}}.\]

La figure \ref{fig:sigma-delta-f-vs-vref} représente un
graphe de cette relation.

\begin{figure}[ht]
	\centering
	\includegraphics[scale=0.6]{img/sigma-delta-f-vs-vref.png}
	\caption{Graphe de la fréquence en fonction de $V_{\text{ref}}$
	pour $\Delta V$ = \unit{0.1}{\volt} et $K = \unit{10}{\second}$.}
	\label{fig:sigma-delta-f-vs-vref}
\end{figure}

On peut faire plusieurs observations à propos de la figure
\ref{fig:sigma-delta-f-vs-vref}. Premièrement, la fréquence
de sortie maximale est atteinte pour $V_{\text{ref}} = \frac{E}{2}$
et vaut
\[ f_{\text{max}} = \frac{K}{\Delta V}\frac{E}{4}. \]
Ensuite, pour un signal $V_{\text{ref}}$ sinusoïdal dont
l'amplitude peut être négative, la fréquence sature.
% FIX ME : je ne sais pas comment exprimer ça correctement.
Or, dans le cas de notre synthétiseur, le signal $V_{\text{ref}}$
est la sortie de notre VCO (après passage dans un filtre pour
en extraire une sinusoïdale pure).
Il faudra donc ``déplacer'' la courbe de la figure
\ref{fig:sigma-delta-f-vs-vref} de manière à ce qu'elle soit
centré autour de l'origine.

\section{Dimensionnement et circuit réel}

% little trick to replace footer.tex by an empty file