\chapter{Modulateur sigma-delta}
\label{sec:sigma-delta}
Le modulateur sigma-delta se trouve à la suite du filtre et avant l'amplificateur de classe D. 
Ce bloc a pour but de moduler le signal d'entrée (un pseudo sinus) par largeur d'impulsion,
c'est-à-dire en signal carré de valeurs seuils constantes mais de fréquence variable dont la
valeur moyenne est égale à la valeur absolue du signal initial. L'obtention d'un signal composé
de deux états est nécessaire pour une utilisation optimale en terme de puissance de l'amplificateur de classe D.

\begin{figure}[ht]
	\centering
	\includegraphics[scale=0.7]{img-sigma-delta/in-out.png}
	\caption{Entré et sortie du sigma-delta}
	\label{fig:entrée-sortie-sigma-delta}
\end{figure}

\section{Fonctionnement du système}
La fréquence de commutation du sigma-delta est supposée très élevée par rapport
au signal d'entrée. Ceci permet de considérer le signal d'entrée constant localement.

Voici ci-dessous le schéma bloc du sigma-delta.

\begin{figure}[ht]
	\centering
	\includegraphics[scale=0.7]{img-sigma-delta/schema-blocs.png}
	\caption{Schéma bloc du modulateur sigma-delta (source : Bruno Dehez).}
	\label{fig:sigma-delta-schema-blocs}
\end{figure}

Ce bloc est constitué de deux parties distinctes : un intégrateur et une bascule asymétrique.
L'intégrateur est caractérisé par un coefficient d'intégration K. La bascule est elle caractérisée
par deux tensions de basculement : $V_H$, la tension de basculement haute fixée arbitrairement à 0 
et $V_L$, la tension de basculement basse avec $\Delta V = V_H - V_L$, ainsi que des valeurs seuils
de sortie : $V_{\text{outH}} = E$, la tension de sortie haute, et la tension $V_{\text{outL}} = 0$. 
$V_{\text{outL}}$ est fixée à 0 car l'amplificateur de classe D nécessite en entré une tension positive ($E$), ou 0.

La période d'oscillation du signal de sortie (qui est 
identique à la période d'oscillation du signal intermédiaire
$V_I$ sur la figure \ref{fig:sigma-delta-schema-blocs})
se calcule en effectuant le raisonnement suivant.
Initialement, soit $V_{\text{ref}}$ positif et
$V_{\text{out}} = 0$. $V_I$ est alors immédiatement positif
et $V_{\text{out}}$ sature directement à $E$. Comme $V_{\text{ref}}$
est $\leq E$, $V_I$ va maintenant décroître jusqu'à atteindre
$\Delta V$. A ce moment précis, $V_{\text{out}} = 0$
et donc $V_I$ va croître jusqu'à atteindre 0, et ainsi de suite.

De là s'obtiennent le temps de descente $t_f$ et le temps de montée $t_r$
du signal $V_I$\footnote{Ce signal sera soit un signal triangulaire,
soit un signal en dents de scie, selon la valeur de $V_{\text{ref}}$.}
\[ t_f = -\frac{\Delta V}{(V_{\text{ref}} - E)K},\]
\[ t_r = \frac{\Delta V}{KV_{\text{ref}}}.\]
La période $T$ étant la somme du temps de descente et du temps
de montée, elle est donnée par
\[ T = \frac{\Delta V}{K}\left(\frac{1}{V_{\text{ref}}} - \frac{1}{V_{\text{ref}} - E}\right) \]
et donc finalement
\begin{equation} 
	f = -\frac{K}{\Delta V} \frac{V_{\text{ref}}(V_{\text{ref}}-E)}{E}.
	\label{eq:sigma-delta-frequency}
\end{equation}

\paragraph{Remarque}
A partir du temps de descente et du temps de montée, il est possible
de prouver que la moyenne du signal carré $V_{\text{out}}$
vaut bien $V_{\text{ref}}$. Il suffit de démontrer l'égalité
suivante
\[ \frac{E \cdot t_f + 0 \cdot t_r}{T} = V_{\text{ref}}.\]

La fréquence en fonction de $V_{\text{ref}}$ est donc
une parabole avec une racine en \unit{0}{\volt} et une
racine en \unit{E}{\volt}.

La fréquence de sortie maximale est atteinte pour 
$V_{\text{ref}} = \frac{E}{2}$ et vaut
\[ f_{\text{max}} = \frac{K}{\Delta V}\frac{E}{4}. \]

Le sigma-delta doit remplir certaines spécifications afin de remplir son rôle au sein du système global : 
\begin{enumerate}	
	\item La fréquence de sortie du sigma-delta doit respecter le théorème d’échantillonnage de Shannon, c'est-à-dire
	que la fréquence de sortie doit être au minimum deux fois supérieure à la fréquence maximale du signal d'entrée
	afin de permettre la reconstitution du signal d'entrée ;
	\item Le signal de sortie ne peut pas atteindre de fréquence nulle pour la gamme d'amplitude du signal
	d'entrée afin éviter d'introduire une fréquence nulle (c'est à dire une constante) en sortie. Il faut donc
	déplacer la parabole de l'équation \ref{eq:sigma-delta-frequency} afin de la centrer en $0$ et de fixer
	ses racines $\pm$\unit{15}{\volt}. Ainsi, la gamme d'amplitude de tension d'entrée étant de $\approx \pm$ \unit{2}{\volt}
	(voir chapitre \ref{sec:filtre}), la fréquence de sortie n'est jamais nulle. De plus, la fréquence maximale du signal 
	d'entrée étant de \unit{8}{\kilo\hertz}, la fréquence du signal de sortie est bien supérieure au double de cette 
	fréquence pour l'ensemble de la gamme d'amplitude, comme le montre la figure \ref{fig:sigma-delta-lim}.
	La condition précédente est donc également bien respectée ;
	\item La fréquence maximale d'entrée de l'amplificateur de classe D est de \unit{100}{\kilo\hertz}.
	En prenant une sécurité de \unit{20}{\kilo\hertz}, la fréquence maximale de sortie est fixée à \unit{80}{\kilo\hertz},
	de telle sorte que :
	\begin{equation} 
		\frac{K}{\Delta V}\frac{E}{4} = 80000.
		\label{eq:sd-f-max}
	\end{equation}
	\item La bascule ne doit pas être sensible au bruit, $\Delta V$ ne doit donc pas être trop faible ;
\end{enumerate}
 
\begin{figure}[ht]
	\centering
	\includegraphics[scale=0.5]{img-sigma-delta/sigma-delta-lim.png}
	\caption{Fréquence de sortie en fonction de la tension d'entrée et contraintes. Cette figure
	met en évidence la limite induite par le théorème de Shannon ainsi que la gamme utilisée par notre système.
	Il est possible de constater que la zone utilisée est nettement inférieure à la zone exploitable. 
	Cela sera mis plus en avant dans la section \ref{sec:pistes-amelioration}.}
	\label{fig:sigma-delta-lim}
\end{figure}

\section{Dimensionnement et circuit réel}
Le circuit du modulateur sigma-delta est représenté
à la figure \ref{fig:sigma-delta-circuit}.

\begin{figure}[ht]
	\centering
	\includegraphics[scale=0.65]{img-sigma-delta/sigma-delta-circuit.png}
	\caption{Circuit du modulateur.}
	\label{fig:sigma-delta-circuit}
\end{figure}

La résolution de ce circuit permet d'obtenir des équations
de la même forme que celles de la figure
\ref{fig:sigma-delta-schema-blocs}.
L'amplificateur opérationnel étant connecté en contre-réaction,
sa bornée d'entrée est virtuellement à la masse : $v_- = v_+ = 0$.
Les différents courants dans le circuit sont alors donnés par
\[ i_{R_4} = \frac{V_{\text{ref}}}{R_4},\]
\[ i_{R_6} = \frac{V_{\text{ee}}}{R_6},\]
\[ i_{R_3} = \frac{V_{\text{out}}}{R_3},\]
\[ i_{C_1} = -C_1\fdif{v_{\text{in}}}{t}.\]
KCL permet ensuite d'écrire l'équation suivante
\[ i_{C_1} = i_{R_4} + i_{R_6} + i_{R_3}\]
et donc d'obtenir
\[ v_{\text{in}} = -\frac{1}{C_1}\int \frac{V_{\text{ref}}}{R_4}
+ \frac{V_{\text{ee}}}{R_6} + \frac{V_{\text{out}}}{R_3}.\]
Pour se ramener à l'équation de la figure
\ref{fig:sigma-delta-schema-blocs} et donc déterminer $K$
en terme des composants du circuits, on
$V'_{\text{ref}} = -R_3(\frac{V_{\text{ref}}}{R_4}+\frac{V_{\text{ee}}}{R_6})$
pour enfin obtenir
\[ v_{\text{in}} = \frac{1}{C_1R_3} \int V'_{\text{ref}} - V_{\text{out}}\]
où $V'_{\text{ref}}$ correspond au $V_{\text{ref}}$
de la figure \ref{fig:sigma-delta-schema-blocs}. Il apparaît alors
que $K = \frac{1}{C_1R_3}$.

Pour centrer la parabole, il faut que $\frac{R_3}{R_6}V_{\text{ee}}$
soit égale à \unit{6.75}{\volt} (c'est à dire $\fra{E}{2}$). Il faut ensuite
étirer la parabole de manière à ce que ses racines soient $\pm$\unit{15}{\volt}.
Il faut donc $\frac{R_3}{R_4} = 0.45$. 

$R_3 =$ \unit{10}{\kilo\ohm} et $R_4 = R_6 =$ \unit{22.220}{\kilo\ohm} sont des valeurs
satisfaisant ces deux équations.

Enfin, il reste l'équation \ref{eq:sd-f-max} fixant la fréquence maximale à résoudre.
Les valeurs suivantes résolvent cette équation et respecte également la contrainte
sur $\Delta V$ : $C_1 = $\unit{1.8}{\nano\farad}, $R_4 = R_6 = $\unit{22.220}{\kilo\ohm}
, $R_1 = $\unit{17.361}{\kilo\ohm} et $R_2 = $\unit{100}{\kilo\ohm}. 

Pour appliquer la signal de sortie du modulateur
à l'étage suivant du circuit, il faudra utiliser un diviseur
résistif car l'étage suivant (l'amplificateur de classe D)
ne supporte pas des entrées supérieures
à \unit{5}{\volt}.

\section{Confrontation des mesures et de la théorie}
En superposant le graphe théorique que nous pouvons obtenir avec les valeurs
obtenues dans la section précédente
et des mesures effectuées sur une implémentation en circuit
du modulateur, nous obtenons la figure \ref{fig:sigma-delta-f-vs-vref-dim-vs-real}.

\begin{figure}[ht]
	\centering
	\includegraphics[scale=0.70]{img-sigma-delta/sigma-delta-f-vs-vref-dim-vs-real.png}
	\caption{En bleu, les prévisions théoriques et en vert les mesures.}
	\label{fig:sigma-delta-f-vs-vref-dim-vs-real}
\end{figure}

Cette figure révèle que la théorie colle assez bien à la réalité. Le
faible décalage dépend sans doute des tolérances des résistances,
des variations dans les alimentations ou encore de la précision des
circuits intégrés.